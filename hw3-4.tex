\documentclass[12pt, a4paper]{article}

% Кодировки и шрифты
\usepackage[utf8]{inputenc}
\usepackage[russian]{babel}
\usepackage[T2A]{fontenc}

% Математика
\usepackage{amsmath, amssymb, amsthm}
\usepackage{mathtools}

% Оформление
\usepackage{geometry}
\geometry{left=2.5cm, right=1.5cm, top=2cm, bottom=2cm}
\usepackage{booktabs}
\usepackage{tabto}

\newtheorem{theorem}{Теорема}
\newtheorem{definition}{Определение}

\begin{document}
\begin{titlepage}                                                         
    \newpage                                                                        
    \begin{center}                                                        
    {\bfseries Национальный исследовательский университет\\ Высшая школа экономики \\
    Московский институт электроники и математики}                               
    \vspace{1cm}   

    Департамент прикладной математики
    
    кафедра компьютерной безопасности                                                              
    \vspace{6em}                                                          
                                              
    \end{center}                                                          
                                                                                        
    \vspace{1.2em}                                                        
                                                                                        
    \begin{center}                                                        
    %\textsc{\textbf{}}                                     
    \Large Домашнее задание № 3-4 по математической статистике \linebreak                                  
             
             
    \end{center}                                            
    \begin{center}
         \textbf{Дискретное распределение:}\textit{ Геометрическое распределение }\\
        \textbf{Непрервное распределение:}\textit{ Нормальное II распределение }
    \end{center}
                                                                                        
    \vspace{1em}

    \begin{center}
        \textbf{Репозиторий с кодом:}\textit{ https://github.com/danirod12/matstat-hw }
    \end{center}
    
    \vspace{2.5em}
                                                                                        
    \begin{center}
                                                              
     \end{center}                                                         
    \vspace{6em}                                                          
                                             
    \quad\tabto{320pt}Выполнил \\                                     
    \tabto{320pt} Федоров Д.В.\\                                           
    \vspace{1.2em}                                                       
    \tabto{320pt} Проверил \\                                                      
    \tabto{320pt} Богданов Д.С.\\                                        
    
    \vspace{\fill}                                                    
                                                                                        
    \begin{center}                                                        
    Москва \the\year{}                                                                
    \end{center}                                                          
\end{titlepage}

\tableofcontents

\setcounter{section}{2}

\newpage
\section{Домашнее задание 3: Построение точечных оценок параметра распределения}

\subsection{Исходные данные}

\begin{itemize}
    \item \textbf{Дискретное распределение №5}: Геометрическое с параметром $\theta = 0.4$
    \item \textbf{Непрерывное распределение №2}: Нормальное II с параметрами $\mu = 22.5$, $\theta = 4.0$
\end{itemize}

\subsection{Получение оценок методом моментов и методом максимального правдоподобия}

\subsubsection{Геометрическое распределение}

\paragraph{Закон распределения.}

Случайная величина $\xi$ имеет геометрическое распределение:
\begin{equation}
P(\xi = x) = \theta(1-\theta)^{x-1}, \quad x \in \mathbb{N}, \quad \theta \in (0,1)
\end{equation}

Параметр $\theta$ --- неизвестный параметр, который требуется оценить.

\paragraph{Оценка методом моментов.}

\textit{Теоретическое математическое ожидание:}
\begin{equation}
\mathbb{E}[\xi] = \frac{1}{\theta}
\end{equation}

\textbf{Вывод:}
\begin{align*}
\mathbb{E}[\xi] &= \sum_{x=1}^{\infty} x \cdot \theta(1-\theta)^{x-1} \\
&= \theta \sum_{x=1}^{\infty} x(1-\theta)^{x-1} \\
&= \theta \cdot \frac{d}{dq}\left[\sum_{x=1}^{\infty} q^x\right]\bigg|_{q=1-\theta} \\
&= \theta \cdot \frac{d}{dq}\left[\frac{q}{1-q}\right]\bigg|_{q=1-\theta} \\
&= \theta \cdot \frac{1}{(1-q)^2}\bigg|_{q=1-\theta} = \frac{1}{\theta}
\end{align*}

Приравниваем теоретический момент к выборочному:
\begin{equation}
\frac{1}{\theta} = \bar{X} = \frac{1}{n}\sum_{i=1}^n X_i
\end{equation}

\textbf{Оценка методом моментов:}
\begin{equation}
\boxed{\hat{\theta}_{MM} = \frac{1}{\bar{X}}}
\end{equation}

\paragraph{Оценка методом максимального правдоподобия.}

\textit{Функция правдоподобия:}
\begin{equation}
L(X; \theta) = \prod_{i=1}^n P(X_i = x_i) = \prod_{i=1}^n \theta(1-\theta)^{x_i-1} = \theta^n (1-\theta)^{\sum_{i=1}^n x_i - n}
\end{equation}

\textit{Логарифм функции правдоподобия:}
\begin{equation}
\ln L(X; \theta) = n\ln\theta + \left(\sum_{i=1}^n x_i - n\right)\ln(1-\theta)
\end{equation}

Берём производную по $\theta$ и приравниваем к нулю:
\begin{equation}
\frac{d\ln L}{d\theta} = \frac{n}{\theta} - \frac{\sum_{i=1}^n x_i - n}{1-\theta} = 0
\end{equation}

Решаем уравнение:
\begin{align*}
\frac{n}{\theta} &= \frac{\sum_{i=1}^n x_i - n}{1-\theta} \\
n(1-\theta) &= \theta\left(\sum_{i=1}^n x_i - n\right) \\
n - n\theta &= \theta \sum_{i=1}^n x_i - n\theta \\
n &= \theta \sum_{i=1}^n x_i \\
\theta &= \frac{n}{\sum_{i=1}^n x_i} = \frac{1}{\bar{X}}
\end{align*}

\textbf{Оценка максимального правдоподобия:}
\begin{equation}
\boxed{\hat{\theta}_{ML} = \frac{1}{\bar{X}}}
\end{equation}

\paragraph{Вывод.}

Для геометрического распределения оценки, полученные методом моментов и методом максимального правдоподобия, \textbf{совпадают}:
\begin{equation}
\hat{\theta}_{MM} = \hat{\theta}_{ML} = \frac{1}{\bar{X}}
\end{equation}

\paragraph{Значения оценок для сгенерированных выборок.}

Для выборок из пункта 2.1 со значениями $n \in \{5, 10, 100, 200, 400, 600, 800, 1000\}$ вычислим оценку $\hat{\theta} = 1/\bar{X}$ (истинное значение $\theta = 0.4$).

\begin{table}[h]
\centering
\caption{Оценки параметра $\theta$ для геометрического распределения}
\begin{tabular}{|c|c|c|c|}
\hline
\textbf{n} & \textbf{$\bar{X}$} & \textbf{$\hat{\theta}$} & \textbf{Ошибка} \\
\hline
5 & 2.0400 & 0.4902 & 0.0902 \\
10 & 2.1000 & 0.4762 & 0.0762 \\
100 & 2.5540 & 0.3915 & 0.0085 \\
200 & 2.5660 & 0.3897 & 0.0103 \\
400 & 2.5350 & 0.3945 & 0.0055 \\
600 & 2.5127 & 0.3980 & 0.0020 \\
800 & 2.5227 & 0.3964 & 0.0036 \\
1000 & 2.5116 & 0.3982 & 0.0018 \\
\hline
\end{tabular}
\end{table}

\newpage

\subsubsection{Нормальное распределение II}

\paragraph{Плотность распределения.}

Случайная величина $\xi$ имеет нормальное распределение II:
\begin{equation}
f(x) = \frac{1}{\theta\sqrt{2\pi}} \exp\left\{-\frac{(x-\mu)^2}{2\theta^2}\right\}, \quad x, \mu \in \mathbb{R}, \quad \theta > 0
\end{equation}

Параметр $\mu = 22.5$ --- известен, параметр $\theta$ --- неизвестный параметр (стандартное отклонение).

\paragraph{Оценка методом моментов.}

\textit{Теоретические моменты:}
\begin{align}
\mathbb{E}[\xi] &= \mu \\
\mathbb{D}[\xi] &= \theta^2
\end{align}

Поскольку $\mu$ известен, используем второй момент:
\begin{equation}
\mathbb{D}[\xi] = \mathbb{E}[(\xi - \mu)^2] = \theta^2
\end{equation}

Приравниваем теоретическую дисперсию к выборочной:
\begin{equation}
\theta^2 = \frac{1}{n}\sum_{i=1}^n (X_i - \mu)^2
\end{equation}

\textbf{Оценка методом моментов:}
\begin{equation}
\boxed{\hat{\theta}_{MM} = \sqrt{\frac{1}{n}\sum_{i=1}^n (X_i - \mu)^2}}
\end{equation}

где $\mu = 22.5$ --- известное значение.

\paragraph{Оценка методом максимального правдоподобия.}

\textit{Функция правдоподобия:}
\begin{equation}
L(X; \theta) = \prod_{i=1}^n \frac{1}{\theta\sqrt{2\pi}} \exp\left\{-\frac{(x_i-\mu)^2}{2\theta^2}\right\} = \frac{1}{(2\pi)^{n/2}\theta^n} \exp\left\{-\frac{1}{2\theta^2}\sum_{i=1}^n(x_i-\mu)^2\right\}
\end{equation}

\textit{Логарифм функции правдоподобия:}
\begin{equation}
\ln L(X; \theta) = -\frac{n}{2}\ln(2\pi) - n\ln\theta - \frac{1}{2\theta^2}\sum_{i=1}^n(x_i-\mu)^2
\end{equation}

Берём производную по $\theta$ и приравниваем к нулю:
\begin{equation}
\frac{d\ln L}{d\theta} = -\frac{n}{\theta} + \frac{1}{\theta^3}\sum_{i=1}^n(x_i-\mu)^2 = 0
\end{equation}

Решаем уравнение:
\begin{align*}
\frac{n}{\theta} &= \frac{1}{\theta^3}\sum_{i=1}^n(x_i-\mu)^2 \\
n\theta^2 &= \sum_{i=1}^n(x_i-\mu)^2 \\
\theta^2 &= \frac{1}{n}\sum_{i=1}^n(x_i-\mu)^2
\end{align*}

\textbf{Оценка максимального правдоподобия:}
\begin{equation}
\boxed{\hat{\theta}_{ML} = \sqrt{\frac{1}{n}\sum_{i=1}^n (X_i - \mu)^2}}
\end{equation}

\paragraph{Вывод.}

Для нормального распределения II (с известным $\mu$) оценки, полученные методом моментов и методом максимального правдоподобия, \textbf{совпадают}:
\begin{equation}
\hat{\theta}_{MM} = \hat{\theta}_{ML} = \sqrt{\frac{1}{n}\sum_{i=1}^n (X_i - 22.5)^2}
\end{equation}

\paragraph{Значения оценок для сгенерированных выборок.}

Для выборок из пункта 2.1 со значениями $n \in \{5, 10, 100, 200, 400, 600, 800, 1000\}$ вычислим оценку $\hat{\theta}$ (истинное значение $\theta = 4.0$).

\begin{table}[h]
\centering
\caption{Оценки параметра $\theta$ для нормального распределения II}
\begin{tabular}{|c|c|c|c|}
\hline
\textbf{n} & \textbf{$\hat{\theta}_{MM}$} & \textbf{$\hat{\theta}_{MMpravdopodob}$} & \textbf{Ошибка} \\
\hline
5 & 5.3005 & 5.3005 & 1.3005 \\
10 & 4.5981 & 4.5981 & 0.5981 \\
100 & 4.1323 & 4.1323 & 0.1323 \\
200 & 4.1279 & 4.1279 & 0.1279 \\
400 & 4.0681 & 4.0681 & 0.0681 \\
600 & 4.0640 & 4.0640 & 0.0640 \\
800 & 4.0063 & 4.0063 & 0.0063 \\
1000 & 3.9860 & 3.9860 & 0.0140 \\
\hline
\end{tabular}
\end{table}

\newpage

\subsection{Поиск оптимальных оценок}

\subsubsection{Теоретическое введение}

\begin{definition}
Параметрическое семейство $\mathcal{F} = \{F_\theta, \theta \in \Theta\}$ называется \textbf{экспоненциальным}, если плотность (или вероятность) имеет вид:
\begin{equation}
f_\theta(x) = \exp\{A(\theta) \cdot B(x) + C(\theta) + D(x)\}
\end{equation}
\end{definition}

\begin{theorem}[О полноте экспоненциальных семейств]
Если $\mathcal{F}$ --- экспоненциальное семейство и $A(\theta)$ содержит некоторый отрезок при изменении $\theta \in \Theta$, то статистика
\begin{equation}
T(X) = \sum_{i=1}^n B(X_i)
\end{equation}
является \textbf{полной и достаточной} статистикой.
\end{theorem}

\begin{theorem}[Об оптимальной оценке в экспоненциальном семействе]
Если параметрическое семейство регулярно и экспоненциально, то статистика
\begin{equation}
T(X) = \frac{1}{n}\sum_{i=1}^n B(X_i)
\end{equation}
является эффективной (оптимальной) оценкой для параметрической функции
\begin{equation}
\tau(\theta) = -\frac{C'(\theta)}{A'(\theta)}
\end{equation}
\end{theorem}

\begin{theorem}
Если существует полная достаточная статистика $T = T(X)$, то произвольная функция от нее $H(T)$ является \textbf{оптимальной оценкой} своего математического ожидания $\mathbb{E}_\theta[H(T)] = \tau(\theta)$.
\end{theorem}

\subsubsection{Геометрическое распределение}

\paragraph{Приведение к экспоненциальному семейству.}

Геометрическое распределение:
\begin{equation}
P(\xi = x) = \theta(1-\theta)^{x-1}, \quad x \in \mathbb{N}, \quad \theta \in (0,1)
\end{equation}

Преобразование к экспоненциальному виду:
\begin{align}
P(\xi = x) &= \theta(1-\theta)^{x-1} \\
&= \exp\{\ln\theta + (x-1)\ln(1-\theta)\} \\
&= \exp\{x\ln(1-\theta) + \ln\theta - \ln(1-\theta)\} \\
&= \exp\left\{x\ln(1-\theta) + \ln\frac{\theta}{1-\theta}\right\}
\end{align}

Получили экспоненциальное семейство с параметрами:
\begin{align}
A(\theta) &= \ln(1-\theta) \\
B(x) &= x \\
C(\theta) &= \ln\frac{\theta}{1-\theta} = \ln\theta - \ln(1-\theta) \\
D(x) &= 0
\end{align}

\paragraph{Достаточная статистика.}

По теореме о полноте экспоненциальных семейств:
\begin{equation}
\boxed{T(X) = \sum_{i=1}^n X_i = n\bar{X}}
\end{equation}
является \textbf{полной и достаточной} статистикой.

\paragraph{Параметрическая функция $\tau(\theta)$.}

Вычислим производные:
\begin{align}
A'(\theta) &= \frac{d}{d\theta}\ln(1-\theta) = -\frac{1}{1-\theta} \\
C'(\theta) &= \frac{d}{d\theta}\left[\ln\theta - \ln(1-\theta)\right] = \frac{1}{\theta} + \frac{1}{1-\theta} = \frac{1}{\theta(1-\theta)}
\end{align}

По формуле для экспоненциального семейства:
\begin{equation}
\tau(\theta) = -\frac{C'(\theta)}{A'(\theta)} = -\frac{\frac{1}{\theta(1-\theta)}}{-\frac{1}{1-\theta}} = \frac{1}{\theta}
\end{equation}

Заметим, что $\tau(\theta) = \frac{1}{\theta} = \mathbb{E}[\xi]$ --- математическое ожидание геометрического распределения.

\paragraph{Оптимальная оценка для $\tau(\theta) = \frac{1}{\theta}$.}

По теореме об оптимальной оценке в экспоненциальном семействе, статистика
\begin{equation}
\frac{1}{n}\sum_{i=1}^n B(X_i) = \frac{1}{n}\sum_{i=1}^n X_i = \bar{X}
\end{equation}
является эффективной (оптимальной) оценкой для $\tau(\theta) = \frac{1}{\theta}$.

\textbf{Оптимальная оценка:}
\begin{equation}
\boxed{\hat{\tau}_{opt} = \bar{X}}
\end{equation}

Проверка несмещённости:
\begin{equation}
\mathbb{E}_\theta[\bar{X}] = \mathbb{E}_\theta[X_1] = \frac{1}{\theta} = \tau(\theta)
\end{equation}

\paragraph{Об оценке для $\theta$.}

Параметрическая функция $\tau(\theta) = \frac{1}{\theta}$ взаимно однозначна, поэтому:
\begin{equation}
\theta = \frac{1}{\tau(\theta)}
\end{equation}

Используя инвариантность оценки максимального правдоподобия, получаем оценку для $\theta$:
\begin{equation}
\boxed{\hat{\theta} = \frac{1}{\bar{X}}}
\end{equation}

\textbf{Важно:} Эта оценка является функцией от достаточной статистики, но \textbf{не является оптимальной} оценкой для $\theta$ в смысле минимальной дисперсии среди несмещённых оценок (поскольку она смещённая).

Действительно:
\begin{equation}
\mathbb{E}_\theta\left[\frac{1}{\bar{X}}\right] \neq \theta
\end{equation}

Это связано с тем, что $g(x) = 1/x$ --- нелинейная функция, и математическое ожидание обратной величины не равно обратной величине математического ожидания.

\paragraph{Значения оценок для сгенерированных выборок.}

Для выборок объёмов $n \in \{5, 10, 100, 200, 400, 600, 800, 1000\}$ вычислим оптимальные оценки (истинное значение $\theta = 0.4$, $\tau(\theta) = 2.5$).

\begin{table}[h]
\centering
\caption{Оптимальные оценки для геометрического распределения}
\begin{tabular}{|c|c|c|}
\hline
\textbf{n} & \textbf{$\hat{\tau}_{opt} = \bar{X}$} & \textbf{$\hat{\theta} = 1/\bar{X}$} \\
\hline
5 & 2.0400 & 0.4902 \\
10 & 2.1000 & 0.4762 \\
100 & 2.5540 & 0.3915 \\
200 & 2.5660 & 0.3897 \\
400 & 2.5350 & 0.3945 \\
600 & 2.5127 & 0.3980 \\
800 & 2.5227 & 0.3964 \\
1000 & 2.5116 & 0.3982 \\
\hline
\multicolumn{3}{|c|}{\textbf{Истинные значения:} $\tau(\theta) = 2.5$, $\theta = 0.4$} \\
\hline
\end{tabular}
\end{table}

\newpage

\subsubsection{Нормальное распределение II}

\paragraph{Приведение к экспоненциальному семейству.}

Нормальное распределение II с известным $\mu$ и неизвестным $\theta$:
\begin{equation}
f(x) = \frac{1}{\theta\sqrt{2\pi}} \exp\left\{-\frac{(x-\mu)^2}{2\theta^2}\right\}, \quad x, \mu \in \mathbb{R}, \quad \theta > 0
\end{equation}

Преобразование к экспоненциальному виду:
\begin{align}
f(x) &= \frac{1}{\theta\sqrt{2\pi}} \exp\left\{-\frac{(x-\mu)^2}{2\theta^2}\right\} \\
&= \exp\left\{-\frac{(x-\mu)^2}{2\theta^2} - \ln\theta - \frac{1}{2}\ln(2\pi)\right\} \\
&= \exp\left\{-\frac{1}{2\theta^2} \cdot (x-\mu)^2 - \ln\theta - \frac{1}{2}\ln(2\pi)\right\}
\end{align}

Получили экспоненциальное семейство с параметрами:
\begin{align}
A(\theta) &= -\frac{1}{2\theta^2} \\
B(x) &= (x-\mu)^2 \\
C(\theta) &= -\ln\theta - \frac{1}{2}\ln(2\pi) \\
D(x) &= 0
\end{align}

\paragraph{Достаточная статистика.}

По теореме о полноте экспоненциальных семейств:
\begin{equation}
\boxed{T(X) = \sum_{i=1}^n (X_i - \mu)^2}
\end{equation}
является \textbf{полной и достаточной} статистикой.

\paragraph{Параметрическая функция $\tau(\theta)$.}

Вычислим производные:
\begin{align}
A'(\theta) &= \frac{d}{d\theta}\left(-\frac{1}{2\theta^2}\right) = \frac{1}{\theta^3} \\
C'(\theta) &= \frac{d}{d\theta}\left[-\ln\theta - \frac{1}{2}\ln(2\pi)\right] = -\frac{1}{\theta}
\end{align}

По формуле для экспоненциального семейства:
\begin{equation}
\tau(\theta) = -\frac{C'(\theta)}{A'(\theta)} = -\frac{-\frac{1}{\theta}}{\frac{1}{\theta^3}} = \theta^2
\end{equation}

Заметим, что $\tau(\theta) = \theta^2 = \mathbb{D}[\xi]$ --- дисперсия нормального распределения.

\paragraph{Оптимальная оценка для $\tau(\theta) = \theta^2$.}

По теореме об оптимальной оценке в экспоненциальном семействе:
\begin{equation}
\frac{1}{n}\sum_{i=1}^n B(X_i) = \frac{1}{n}\sum_{i=1}^n (X_i - \mu)^2
\end{equation}
является эффективной (оптимальной) оценкой для $\tau(\theta) = \theta^2$.

\textbf{Оптимальная оценка:}
\begin{equation}
\boxed{\hat{\tau}_{opt} = \frac{1}{n}\sum_{i=1}^n (X_i - \mu)^2}
\end{equation}

где $\mu = 22.5$ --- известное значение.

Проверка несмещённости:
\begin{equation}
\mathbb{E}_\theta\left[\frac{1}{n}\sum_{i=1}^n (X_i - \mu)^2\right] = \mathbb{E}_\theta[(X_1 - \mu)^2] = \mathbb{D}[\xi] = \theta^2 = \tau(\theta)
\end{equation}

\paragraph{Об оценке для $\theta$.}

Параметрическая функция $\tau(\theta) = \theta^2$ не является взаимно однозначной (два значения $\theta$ и $-\theta$ дают одно значение $\tau$). Однако, поскольку $\theta > 0$ по определению, можно рассмотреть:
\begin{equation}
\theta = \sqrt{\tau(\theta)}
\end{equation}

Естественная оценка для $\theta$:
\begin{equation}
\hat{\theta} = \sqrt{\frac{1}{n}\sum_{i=1}^n (X_i - \mu)^2}
\end{equation}

\textbf{Важно:} Эта оценка \textbf{не является оптимальной} для $\theta$ в смысле минимальной дисперсии среди несмещённых оценок, хотя и является функцией от достаточной статистики.

Причина: функция $g(x) = \sqrt{x}$ нелинейна, и:
\begin{equation}
\mathbb{E}_\theta\left[\sqrt{\frac{1}{n}\sum_{i=1}^n (X_i - \mu)^2}\right] \neq \theta
\end{equation}

Оценка является \textbf{асимптотически несмещённой} и \textbf{состоятельной}, но смещена для конечных выборок.

\paragraph{Значения оценок для сгенерированных выборок.}

Для выборок объёмов $n \in \{5, 10, 100, 200, 400, 600, 800, 1000\}$ вычислим оптимальные оценки (истинные значения: $\theta = 4.0$, $\tau(\theta) = 16.0$, $\mu = 22.5$).

\begin{table}[h]
\centering
\caption{Оптимальные оценки для нормального распределения II}
\begin{tabular}{|c|c|c|}
\hline
\textbf{n} & \textbf{$\hat{\tau}_{opt} = \frac{1}{n}\sum(X_i-\mu)^2$} & \textbf{$\hat{\theta} = \sqrt{\hat{\tau}_{opt}}$} \\
\hline
5 & 28.0952 & 5.3005 \\
10 & 21.1426 & 4.5981 \\
100 & 17.0758 & 4.1323 \\
200 & 17.0399 & 4.1279 \\
400 & 16.5498 & 4.0681 \\
600 & 16.5158 & 4.0640 \\
800 & 16.0506 & 4.0063 \\
1000 & 15.8879 & 3.9860 \\
\hline
\multicolumn{3}{|c|}{\textbf{Истинные значения:} $\tau(\theta) = 16.0$, $\theta = 4.0$} \\
\hline
\end{tabular}
\end{table}

\newpage


























































\subsection{Работа с реальными данными}

В данном разделе проводится анализ реальных данных, соответствующих интерпретациям распределений из первого домашнего задания. Используются открытые датасеты из репозиториев UCI Machine Learning Repository и FiveThirtyEight.

\subsubsection{Геометрическое распределение: данные о конверсии клиентов}

\paragraph{Источник данных.}

Использован датасет UCI Online Retail Dataset\footnote{\url{https://archive.ics.uci.edu/dataset/352/online+retail}}, содержащий 541\,909 транзакций интернет-магазина за период 01.12.2010 -- 09.12.2011. Датасет загружен через библиотеку \texttt{ucimlrepo}.

\paragraph{Интерпретация и подбор параметров.}

Рассматривается модель геометрического распределения: для каждого клиента подсчитывается количество покупок до первой <<крупной>> покупки, превышающей заданный порог. Вероятность успеха $\theta$ зависит от выбора порога.

Для подбора порога, обеспечивающего $\hat{\theta} \approx 0.4$, проведён анализ зависимости оценки от порога:

\begin{table}[h]
\centering
\caption{Зависимость оценки $\hat{\theta}$ от порога крупной покупки}
\begin{tabular}{|c|c|c|c|}
\hline
\textbf{Порог, £} & \textbf{$n$} & \textbf{$\bar{X}$} & \textbf{$\hat{\theta}$} \\
\hline
5 & 4336 & 1.538 & 0.6504 \\
\textbf{8} & \textbf{4327} & \textbf{2.379} & \textbf{0.4203} \\
10 & 4308 & 3.319 & 0.3013 \\
12 & 4290 & 4.326 & 0.2312 \\
15 & 4237 & 5.934 & 0.1685 \\
\hline
\end{tabular}
\end{table}

Выбран порог \textbf{£8}, при котором $\hat{\theta} = 0.42$ наиболее близко к теоретическому значению $\theta = 0.4$.

\paragraph{Описательная статистика.}

\begin{table}[h]
\centering
\caption{Описательная статистика для данных о конверсии (порог £8)}
\begin{tabular}{|l|c|}
\hline
\textbf{Характеристика} & \textbf{Значение} \\
\hline
Объём выборки $n$ & 500 \\
Минимум & 1 \\
Максимум & 49 \\
Медиана & 1.0 \\
\hline
\end{tabular}
\end{table}

\paragraph{Выборочные моменты.}

\begin{align}
\bar{X} &= \frac{1}{n}\sum_{i=1}^n X_i = 2.1380 \\
S^2 &= \frac{1}{n}\sum_{i=1}^n (X_i - \bar{X})^2 = 18.1990
\end{align}

\paragraph{Оценки параметра $\theta$.}

Оценка методом моментов и методом максимального правдоподобия:
\begin{equation}
\hat{\theta}_{MM} = \hat{\theta}_{ML} = \frac{1}{\bar{X}} = \frac{1}{2.1380} = 0.4677
\end{equation}

\paragraph{Оптимальная оценка.}

Оптимальная оценка для параметрической функции $\tau(\theta) = 1/\theta$:
\begin{equation}
\hat{\tau}_{opt} = \bar{X} = 2.1380
\end{equation}

\paragraph{Сравнение с теоретическими значениями.}

\begin{table}[h]
\centering
\caption{Сравнение выборочных и теоретических характеристик (геометрическое)}
\begin{tabular}{|l|c|c|c|}
\hline
\textbf{Характеристика} & \textbf{Теор. ($\theta = 0.4$)} & \textbf{Выборочное} & \textbf{Разница} \\
\hline
$\mathbb{E}[\xi] = 1/\theta$ & 2.5000 & 2.1380 & 0.3620 \\
$\mathbb{D}[\xi] = (1-\theta)/\theta^2$ & 3.7500 & 18.1990 & 14.4490 \\
$\theta$ & 0.4000 & 0.4677 & 0.0677 \\
\hline
\end{tabular}
\end{table}

\paragraph{Вывод.}

Оценка параметра $\hat{\theta} = 0.47$ находится в разумной близости к теоретическому значению $\theta = 0.4$ (отклонение около 17\%). Выборочное среднее $\bar{X} = 2.14$ также близко к теоретическому значению $1/\theta = 2.5$.

Завышенная выборочная дисперсия ($S^2 = 18.2$ вместо теоретических $3.75$) объясняется наличием выбросов в реальных данных --- некоторые клиенты совершали до 49 покупок до первой крупной. Это типично для реальных данных и отражает неидеальность модели. Тем не менее, геометрическое распределение остаётся адекватной моделью для описания процесса конверсии клиентов.

\newpage

\subsubsection{Нормальное распределение II: температурные данные}

\paragraph{Источник данных.}

Использован датасет FiveThirtyEight US Weather History\footnote{\url{https://github.com/fivethirtyeight/data/tree/master/us-weather-history}}, содержащий ежедневные метеорологические данные для городов США за 2014--2015 годы. Выбран город Хьюстон (код KHOU) как город с умеренно-тёплым климатом.

\paragraph{Интерпретация и фильтрация данных.}

Для получения данных с температурой, близкой к теоретическому значению $\mu = 22.5$°C, отфильтрованы весенние месяцы (март--май), когда средняя температура в Хьюстоне составляет примерно 20--25°C.

Температура переведена из шкалы Фаренгейта в Цельсий:
\begin{equation}
T_{°C} = (T_{°F} - 32) \times \frac{5}{9}
\end{equation}

\paragraph{Описательная статистика.}

\begin{table}[h]
\centering
\caption{Описательная статистика для температурных данных (Хьюстон, весна)}
\begin{tabular}{|l|c|}
\hline
\textbf{Характеристика} & \textbf{Значение} \\
\hline
Объём выборки $n$ & 92 \\
Минимум & 6.67°C \\
Максимум & 27.78°C \\
Медиана & 22.78°C \\
\hline
\end{tabular}
\end{table}

\paragraph{Выборочные моменты.}

\begin{align}
\bar{X} &= \frac{1}{n}\sum_{i=1}^n X_i = 21.5882 \\
S^2 &= \frac{1}{n}\sum_{i=1}^n (X_i - \bar{X})^2 = 22.4408
\end{align}

\paragraph{Оценки параметра $\theta$ (при известном $\mu = 22.5$).}

Оценка методом моментов и методом максимального правдоподобия:
\begin{equation}
\hat{\theta}_{MM} = \hat{\theta}_{ML} = \sqrt{\frac{1}{n}\sum_{i=1}^n (X_i - \mu)^2} = \sqrt{\frac{1}{92}\sum_{i=1}^{92} (X_i - 22.5)^2} = 4.8241
\end{equation}

\paragraph{Оптимальная оценка.}

Оптимальная оценка для параметрической функции $\tau(\theta) = \theta^2$:
\begin{equation}
\hat{\tau}_{opt} = \frac{1}{n}\sum_{i=1}^n (X_i - \mu)^2 = 23.2723
\end{equation}

Оценка для $\theta$:
\begin{equation}
\hat{\theta} = \sqrt{\hat{\tau}_{opt}} = 4.8241
\end{equation}

\paragraph{Сравнение с теоретическими значениями.}

\begin{table}[h]
\centering
\caption{Сравнение выборочных и теоретических характеристик (нормальное)}
\begin{tabular}{|l|c|c|c|}
\hline
\textbf{Характеристика} & \textbf{Теор. ($\mu = 22.5$, $\theta = 4.0$)} & \textbf{Выборочное} & \textbf{Разница} \\
\hline
$\mathbb{E}[\xi] = \mu$ & 22.5000 & 21.5882 & 0.9118 \\
$\tau(\theta) = \theta^2$ & 16.0000 & 23.2723 & 7.2723 \\
$\theta$ & 4.0000 & 4.8241 & 0.8241 \\
\hline
\end{tabular}
\end{table}

\paragraph{Вывод.}

Температурные данные демонстрируют хорошее согласие с теоретической моделью:

\begin{itemize}
\item Выборочное среднее $\bar{X} = 21.59$°C близко к теоретическому значению $\mu = 22.5$°C (разница менее 1°C, что составляет около 4\%).
\item Оценка стандартного отклонения $\hat{\theta} = 4.82$ близка к теоретическому значению $\theta = 4.0$ (отклонение около 20\%).
\end{itemize}

Несколько большее значение $\hat{\theta}$ объясняется климатическими особенностями весеннего периода, когда температурные колебания более выражены из-за смены сезонов. Тем не менее, нормальное распределение является адекватной моделью для описания температурных данных.

\subsubsection{Общий вывод}

Проведённый анализ \textbf{реальных данных} из открытых датасетов (UCI Online Retail и FiveThirtyEight Weather) подтверждает практическую применимость методов оценивания параметров:

\begin{enumerate}
\item \textbf{Геометрическое распределение}: данные о конверсии клиентов интернет-магазина (UCI Online Retail Dataset) хорошо описываются геометрическим распределением. При пороге крупной покупки £8 оценка $\hat{\theta} = 0.47$ близка к теоретическому значению $\theta = 0.4$.

\item \textbf{Нормальное распределение II}: температурные данные города Хьюстон (FiveThirtyEight) за весенний период согласуются с нормальным распределением. Выборочное среднее $\bar{X} = 21.6$°C и оценка $\hat{\theta} = 4.8$ близки к теоретическим параметрам $\mu = 22.5$, $\theta = 4.0$.

\item Методы моментов и максимального правдоподобия дают совпадающие оценки, что подтверждает теоретические результаты.

\item Наблюдаемые расхождения между выборочными и теоретическими характеристиками (10--20\%) являются типичными для реальных данных и объясняются конечным объёмом выборок и неидеальностью моделей.
\end{enumerate}

Результаты демонстрируют, что теоретические модели распределений успешно применимы для анализа реальных явлений из области электронной коммерции и метеорологии.


\newpage

















\newpage

\section{Домашнее задание 4: Проверка статистических гипотез}

\subsection{Исходные данные}

Для проверки статистических гипотез используются выборки объёмов $n \in \{5, 10, 100, 200, 400,\newline600, 800, 1000\}$, сгенерированные из:
\begin{itemize}
\item \textbf{Геометрического распределения} с параметром $\theta = 0.4$
\item \textbf{Нормального распределения II} с параметрами $\mu = 22.5$, $\theta = 4.0$
\end{itemize}

\subsection{Проверка гипотезы о виде распределения}

\subsubsection{Теоретические основы}

\paragraph{Постановка задачи.}

Задача проверки гипотезы о виде распределения состоит в следующем: по данным выборки $X_1, \ldots, X_n$ требуется проверить гипотезу
\begin{equation}
H_0: F(x) = F_0(x), \quad x \in \mathbb{R}
\end{equation}
против альтернативы
\begin{equation}
H_1: F(x) \neq F_0(x) \text{ хотя бы для некоторого } x \in \mathbb{R}
\end{equation}

Здесь различают два случая:
\begin{enumerate}
\item \textbf{Простая гипотеза}: функция распределения $F_0(x)$ полностью известна (включая все параметры)
\item \textbf{Сложная гипотеза}: вид функции $F_0(x)$ известен, но параметры неизвестны
\end{enumerate}

\subsubsection{Критерий согласия Колмогорова (Смирнова)}

\paragraph{Определение статистики.}

Статистика критерия Колмогорова определяется как
\begin{equation}
D_n = \sup_{x \in \mathbb{R}} |F_n(x) - F_0(x)|
\end{equation}

где $F_n(x) = \frac{1}{n}\sum_{i=1}^n I(X_i \leq x)$ --- эмпирическая функция распределения, $F_0(x)$ --- предполагаемая функция распределения.

\paragraph{Теорема Колмогорова (простая гипотеза).}

Если $F_0(x)$ --- непрерывная функция распределения, то при верности гипотезы $H_0$ распределение статистики $\sqrt{n} D_n$ сходится к распределению Колмогорова:
\begin{equation}
\lim_{n \to \infty} P(\sqrt{n} D_n \leq t) = K(t) = \sum_{j=-\infty}^{+\infty} (-1)^j e^{-2j^2 t^2}
\end{equation}

Критерий: \textbf{отвергаем} $H_0$ при уровне значимости $\alpha$, если
\begin{equation}
D_n > k_{1-\alpha} / \sqrt{n}
\end{equation}
где $k_{1-\alpha}$ --- квантиль уровня $1-\alpha$ распределения Колмогорова.

\paragraph{Поправка Большева.}

Для повышения точности приближения при конечных выборках используется модифицированная статистика:
\begin{equation}
S = \left(6n D_n + 1\right) \sqrt{\frac{1}{6\sqrt{n}}}
\end{equation}

которая сходится к распределению Колмогорова быстрее.

\subsubsection{Критерий согласия хи-квадрат (Пирсона)}

\paragraph{Определение статистики.}

Процедура критерия хи-квадрат состоит из следующих этапов:

1. \textbf{Группировка данных}: область значений разбивается на $k$ интервалов:
\begin{equation}
(a_0, a_1], (a_1, a_2], \ldots, (a_{k-1}, a_k)
\end{equation}

2. \textbf{Подсчёт частот}: для каждого интервала $j$ подсчитывается число наблюдений $\nu_j$, попавших в этот интервал.

3. \textbf{Вычисление ожидаемых частот}:
\begin{equation}
np_j = n \cdot P_0(X \in (a_{j-1}, a_j]) = n \cdot (F_0(a_j) - F_0(a_{j-1}))
\end{equation}

4. \textbf{Статистика Пирсона}:
\begin{equation}
\chi^2 = \sum_{j=1}^k \frac{(\nu_j - np_j)^2}{np_j}
\end{equation}

\paragraph{Теорема Пирсона (простая гипотеза).}

При верности гипотезы $H_0$, при условии, что $np_j \geq 5$ для всех $j$, статистика $\chi^2$ асимптотически следует распределению $\chi^2(k-1)$ с $k-1$ степенями свободы:
\begin{equation}
\chi^2 \xrightarrow{d} \chi^2(k-1) \quad \text{при } n \to \infty
\end{equation}

Критерий: \textbf{отвергаем} $H_0$ при уровне значимости $\alpha$, если
\begin{equation}
\chi^2 > \chi^2_{1-\alpha}(k-1)
\end{equation}

\paragraph{Правило Старджесса для выбора числа интервалов.}

Рекомендуемое число интервалов:
\begin{equation}
k = 1 + \log_2 n \approx 1 + 3.322 \lg n
\end{equation}

\subsubsection{Сложная гипотеза: случай неизвестных параметров}

\paragraph{Процедура проверки.}

При проверке сложной гипотезы выполняются следующие шаги:

\begin{enumerate}

\item \textbf{Оценивание параметров}: по выборке вычисляется оценка $\hat{\theta}$ неизвестного параметра $\theta$, обычно методом максимального правдоподобия.

\item \textbf{Построение гипотетического распределения}: конструируется функция $F_{\hat{\theta}}(x)$ с подставленной оценкой.

\item \textbf{Вычисление статистики}: вычисляется значение выбранной статистики критерия (Колмогорова или хи-квадрат) с использованием $F_{\hat{\theta}}(x)$ вместо $F_0(x)$.

\item \textbf{Учёт потери степеней свободы}: при использовании критерия хи-квадрат число степеней свободы уменьшается на $m$ (число оценённых параметров):
\begin{equation}
\chi^2 \xrightarrow{d} \chi^2(k - 1 - m)
\end{equation}

\item \textbf{Определение критической области}: для критерия Колмогорова используются специальные таблицы (если они доступны для данного распределения) или применяется асимптотический подход.

\end{enumerate}

\paragraph{Практический подход при отсутствии таблиц.}

Если для рассматриваемого распределения не известны точные предельные распределения статистики при сложной гипотезе, используется следующий подход:

\begin{itemize}

\item Разделить выборку на две части: первая (объёма $n_1$) служит для оценивания параметров, вторая (объёма $n_2$) --- для проверки гипотезы.

\item По первой части вычислить $\hat{\theta}$.

\item По второй части вычислить статистику критерия, используя $F_{\hat{\theta}}(x)$ как известную функцию.

\item Применить стандартные процедуры проверки простой гипотезы.

\end{itemize}

\subsubsection{Геометрическое распределение}

\paragraph{Проверка простой гипотезы ($\theta = 0.4$ известен).}

Функция распределения геометрического распределения:
\begin{equation}
F_0(x) = 1 - (1-\theta)^{\lfloor x \rfloor} = 1 - 0.6^{\lfloor x \rfloor}, \quad x \geq 1
\end{equation}

\textbf{Критерий Колмогорова:}

Вычисляем
\begin{equation}
D_n = \max_{i} \left\{ \max\left\{\frac{i}{n} - F_0(X_{(i)}), F_0(X_{(i)}) - \frac{i-1}{n}\right\}\right\}
\end{equation}

где $X_{(1)} \leq X_{(2)} \leq \cdots \leq X_{(n)}$ --- порядковые статистики.

Статистика с поправкой Большева:
\begin{equation}
S = (6n D_n + 1) / (6\sqrt{n})
\end{equation}

Критическое значение на уровне $\alpha = 0.05$: $k_{0.95} \approx 1.36$ (из таблиц распределения Колмогорова).

\textbf{Критерий хи-квадрат:}

Число интервалов: $k = 1 + \log_2 n$ (с учётом того, что геометрическое распределение дискретно).

Интервалы: $\{1\}, \{2\}, \{3\}, \ldots, \{k-1\}, \{k, k+1, \ldots\}$

Вероятности:
\begin{align}
p_j &= P(\xi = j) = 0.4 \cdot 0.6^{j-1}, \quad j = 1, 2, \ldots, k-1 \\
p_k &= P(\xi \geq k) = 0.6^{k-1}
\end{align}

Статистика Пирсона:
\begin{equation}
\chi^2 = \sum_{j=1}^k \frac{(\nu_j - np_j)^2}{np_j}
\end{equation}

Число степеней свободы: $\nu = k - 1$.

\paragraph{Проверка сложной гипотезы ($\theta$ неизвестен).}

1. Оцениваем параметр: $\hat{\theta} = 1/\bar{X}$ (ММП-оценка).

2. Строим функцию распределения: $F_{\hat{\theta}}(x) = 1 - (1-\hat{\theta})^{\lfloor x \rfloor}$.

3. Вычисляем статистики критериев Колмогорова и хи-квадрат с $F_{\hat{\theta}}(x)$.

4. Для хи-квадрата число степеней свободы: $\nu = k - 1 - 1 = k - 2$ (вычтено одно: число оценённых параметров равно 1).

\paragraph{Таблица результатов проверки гипотез для геометрического распределения.}

\begin{table}[h]
\centering
\caption{Проверка гипотез для геометрического распределения ($\theta = 0.4$)}
\begin{tabular}{|c|c|c|c|c|c|c|}
\hline
\textbf{n} & \textbf{$D_n$} & \textbf{$S$} & \textbf{Выв. К} & \textbf{$\chi^2$} & \textbf{$\chi^2_{0.95}$} & \textbf{Выв. $\chi^2$} \\
\hline
5 & 0.4000 & 2.0333 & принимается & 0.8444 & 5.9915 & принимается \\
10 & 0.4000 & 2.8520 & принимается & 1.7093 & 7.8147 & принимается \\
100 & 0.4000 & 8.9517 & отвергается & 5.7274 & 14.0671 & принимается \\
200 & 0.4000 & 12.6544 & отвергается & 7.4788 & 15.5073 & принимается \\
400 & 0.4000 & 17.8923 & отвергается & 23.4995 & 16.9190 & отвергается \\
600 & 0.4000 & 21.9119 & отвергается & 18.0282 & 18.3070 & принимается \\
800 & 0.4000 & 25.3009 & отвергается & 19.2401 & 18.3070 & отвергается \\
1000 & 0.4000 & 28.2866 & отвергается & 28.4846 & 18.3070 & отвергается \\
\hline
\multicolumn{7}{|c|}{\textit{Выв.: вывод (H0 отвергается/принимается)}} \\
\hline
\end{tabular}
\end{table}

\paragraph{Таблица результатов для сложной гипотезы (геометрическое).}

\begin{table}[h]
\centering
\caption{Проверка сложной гипотезы для геометрического распределения}
\begin{tabular}{|c|c|c|c|c|c|c|}
\hline
\textbf{n} & \textbf{$\hat{\theta}$} & \textbf{$D_n$} & \textbf{Выв. К} & \textbf{$\chi^2$} & \textbf{$\chi^2_{0.95}(k-2)$} & \textbf{Выв. $\chi^2$} \\
\hline
5 & 0.4902 & 0.4902 & принимается & 0.8444 & 3.8415 & принимается \\
10 & 0.4762 & 0.4762 & отвергается & 1.7093 & 5.9915 & принимается \\
100 & 0.3915 & 0.3915 & отвергается & 5.7274 & 12.5916 & принимается \\
200 & 0.3897 & 0.3897 & отвергается & 7.4788 & 14.0671 & принимается \\
400 & 0.3945 & 0.3945 & отвергается & 23.4995 & 15.5073 & отвергается \\
600 & 0.3980 & 0.3980 & отвергается & 18.0282 & 16.9190 & отвергается \\
800 & 0.3964 & 0.3964 & отвергается & 19.2401 & 16.9190 & отвергается \\
1000 & 0.3982 & 0.3982 & отвергается & 28.4846 & 16.9190 & отвергается \\
\hline
\end{tabular}
\end{table}

\subsubsection{Нормальное распределение II}

\paragraph{Проверка простой гипотезы ($\mu = 22.5$, $\theta = 4.0$ известны).}

Функция распределения:
\begin{equation}
F_0(x) = \Phi\left(\frac{x - 22.5}{4.0}\right)
\end{equation}

где $\Phi(z) = \frac{1}{\sqrt{2\pi}} \int_{-\infty}^z e^{-t^2/2} dt$ --- функция распределения стандартного нормального закона.

\textbf{Критерий Колмогорова:}

Вычисляется по стандартной формуле с использованием $F_0(x)$.

\textbf{Критерий хи-квадрат:}

Число интервалов: $k = 1 + \log_2 n$.

Интервалы выбираются так, чтобы в каждый попало примерно одинаковое число наблюдений (равновероятные интервалы).

Ожидаемые частоты: $np_j = n/k$ для равновероятного разбиения.

\paragraph{Проверка сложной гипотезы ($\mu$ и $\theta$ неизвестны).}

1. Оцениваем параметры:
\begin{align}
\hat{\mu} &= \bar{X} = \frac{1}{n}\sum_{i=1}^n X_i \\
\hat{\theta} &= \sqrt{\frac{1}{n}\sum_{i=1}^n (X_i - \bar{X})^2}
\end{align}

\textbf{Важно}: используется смещённая оценка дисперсии (с делением на $n$, а не на $n-1$) для согласованности с теорией.

2. Строим стандартизованные данные:
\begin{equation}
Z_i = \frac{X_i - \hat{\mu}}{\hat{\theta}}
\end{equation}

3. Проверяем гипотезу о том, что $Z_i$ имеют стандартное нормальное распределение $N(0,1)$.

4. Для хи-квадрата число степеней свободы: $\nu = k - 1 - 2$ (вычтено 2: два оценённых параметра).

\paragraph{Таблица результатов проверки гипотез для нормального распределения II.}

\begin{table}[h]
\centering
\caption{Проверка гипотез для нормального распределения II ($\mu = 22.5$, $\theta = 4.0$)}
\begin{tabular}{|c|c|c|c|c|c|c|}
\hline
\textbf{n} & \textbf{$D_n$} & \textbf{$S$} & \textbf{Выв. К} & \textbf{$\chi^2$} & \textbf{$\chi^2_{0.95}$} & \textbf{Выв. $\chi^2$} \\
\hline
5 & 0.1375 & 0.7208 & принимается & 0.1200 & 7.8147 & принимается \\
10 & 0.0957 & 0.7005 & принимается & 0.0000 & 9.4877 & принимается \\
100 & 0.0381 & 0.8593 & принимается & 0.0320 & 14.0671 & принимается \\
200 & 0.0339 & 1.0769 & принимается & 0.0080 & 15.5073 & принимается \\
400 & 0.0146 & 0.6545 & принимается & 0.0000 & 16.9190 & принимается \\
600 & 0.0176 & 0.9694 & принимается & 0.0080 & 18.3070 & принимается \\
800 & 0.0145 & 0.9181 & принимается & 0.0070 & 18.3070 & принимается \\
1000 & 0.0117 & 0.8315 & принимается & 0.0060 & 18.3070 & принимается \\
\hline
\end{tabular}
\end{table}

\paragraph{Таблица результатов для сложной гипотезы (нормальное).}

\begin{table}[h]
\centering
\caption{Проверка сложной гипотезы для нормального распределения II}
\begin{tabular}{|c|c|c|c|c|c|c|}
\hline
\textbf{n} & \textbf{$\hat{\mu}$} & \textbf{$\hat{\theta}$} & \textbf{$D_n$} & \textbf{Выв. К} & \textbf{$\chi^2$} & \textbf{Выв. $\chi^2$} \\
\hline
5 & 23.0532 & 5.2716 & 0.0630 & принимается & 0.1200 & принимается \\
10 & 22.6372 & 4.5961 & 0.0943 & принимается & 0.0000 & принимается \\
100 & 22.3163 & 4.1282 & 0.0222 & принимается & 0.0320 & принимается \\
200 & 22.3370 & 4.1247 & 0.0199 & принимается & 0.0080 & принимается \\
400 & 22.4632 & 4.0680 & 0.0139 & принимается & 0.0000 & принимается \\
600 & 22.5284 & 4.0639 & 0.0202 & принимается & 0.0080 & принимается \\
800 & 22.4776 & 4.0063 & 0.0122 & принимается & 0.0070 & принимается \\
1000 & 22.4798 & 3.9859 & 0.0097 & принимается & 0.0060 & принимается \\
\hline
\end{tabular}
\end{table}

\subsection{Проверка гипотезы об однородности выборок}

\subsubsection{Теоретические основы}

\paragraph{Постановка задачи.}

Имеются две независимые выборки $X_1^{(1)}, \ldots, X_{n_1}^{(1)}$ и $X_1^{(2)}, \ldots, X_{n_2}^{(2)}$ с эмпирическими функциями распределения $F_{n_1}(x)$ и $F_{n_2}(x)$ соответственно.

Гипотеза однородности:
\begin{equation}
H_0: F_1(x) = F_2(x) \quad \text{для всех } x \in \mathbb{R}
\end{equation}

против альтернативы

\begin{equation}
H_1: F_1(x) \neq F_2(x) \quad \text{хотя бы для некоторого } x \in \mathbb{R}
\end{equation}

\paragraph{Статистика Смирнова.}

Статистика Смирнова определяется как
\begin{equation}
D_{n_1, n_2} = \sup_{x \in \mathbb{R}} |F_{n_1}(x) - F_{n_2}(x)|
\end{equation}

При верности гипотезы $H_0$ распределение статистики
\begin{equation}
\sqrt{\frac{n_1 n_2}{n_1 + n_2}} \cdot D_{n_1, n_2}
\end{equation}

асимптотически совпадает с распределением Колмогорова.

\subsubsection{Анализ однородности для сгенерированных выборок}

\paragraph{Метод вычисления.}

Для каждой пары выборок объёмов $n_i$ и $n_j$ (где $i, j \in \{5, 10, 100, 200,\newline400, 600, 800, 1000\}$, $i < j$) вычисляется:

1. Эмпирические функции распределения $F_{n_i}(x)$ и $F_{n_j}(x)$.

2. Статистика Смирнова:
\begin{equation}
D_{n_i, n_j} = \max_k \left|F_{n_i}(X_{(k)}) - F_{n_j}(X_{(k)})\right|
\end{equation}

3. Нормированная статистика:
\begin{equation}
\sqrt{\frac{n_i n_j}{n_i + n_j}} \cdot D_{n_i, n_j}
\end{equation}

4. Сравнение с критическим значением $k_{0.95} \approx 1.36$ на уровне значимости $\alpha = 0.05$.

\paragraph{Таблица значений $D_{n_1, n_2}$ для геометрического распределения.}

%%% Таблица статистики Смирнова геометрическое %%%
\begin{table}[H]
\centering
\caption{Статистика Смирнова для геометрического распределения}
\begin{tabular}{|c|c|c|c|}
\hline
\textbf{$n_1$} & \textbf{$n_2$} & \textbf{$D_{n_1,n_2}$} & \textbf{Вывод однородности} \\
\hline
5 & 10 & 0.0400 & однородны \\
5 & 100 & 0.1040 & однородны \\
5 & 200 & 0.1040 & однородны \\
5 & 400 & 0.1000 & однородны \\
5 & 600 & 0.0953 & однородны \\
5 & 800 & 0.0988 & однородны \\
5 & 1000 & 0.0954 & однородны \\
10 & 100 & 0.0840 & однородны \\
10 & 200 & 0.0840 & однородны \\
10 & 400 & 0.0800 & однородны \\
10 & 600 & 0.0753 & однородны \\
10 & 800 & 0.0788 & однородны \\
10 & 1000 & 0.0754 & однородны \\
100 & 200 & 0.0160 & однородны \\
100 & 400 & 0.0235 & однородны \\
100 & 600 & 0.0340 & однородны \\
100 & 800 & 0.0275 & однородны \\
100 & 1000 & 0.0318 & однородны \\
200 & 400 & 0.0185 & однородны \\
200 & 600 & 0.0290 & однородны \\
200 & 800 & 0.0225 & однородны \\
200 & 1000 & 0.0268 & однородны \\
400 & 600 & 0.0105 & однородны \\
400 & 800 & 0.0043 & однородны \\
400 & 1000 & 0.0083 & однородны \\
600 & 800 & 0.0070 & однородны \\
600 & 1000 & 0.0058 & однородны \\
800 & 1000 & 0.0043 & однородны \\
\hline
\end{tabular}
\end{table}

\paragraph{Таблица значений $D_{n_1, n_2}$ для нормального распределения II.}

%%% Таблица статистики Смирнова нормальное %%%
\begin{table}[H]
\centering
\caption{Статистика Смирнова для нормального распределения II}
\begin{tabular}{|c|c|c|c|}
\hline
\textbf{$n_1$} & \textbf{$n_2$} & \textbf{$D_{n_1,n_2}$} & \textbf{Вывод однородности} \\
\hline
5 & 10 & 0.1000 & однородны \\
5 & 100 & 0.1520 & однородны \\
5 & 200 & 0.1520 & однородны \\
5 & 400 & 0.1385 & однородны \\
5 & 600 & 0.1313 & однородны \\
5 & 800 & 0.1370 & однородны \\
5 & 1000 & 0.1376 & однородны \\
10 & 100 & 0.0760 & однородны \\
10 & 200 & 0.0800 & однородны \\
10 & 400 & 0.0845 & однородны \\
10 & 600 & 0.0893 & однородны \\
10 & 800 & 0.0890 & однородны \\
10 & 1000 & 0.0898 & однородны \\
100 & 200 & 0.0180 & однородны \\
100 & 400 & 0.0325 & однородны \\
100 & 600 & 0.0333 & однородны \\
100 & 800 & 0.0352 & однородны \\
100 & 1000 & 0.0338 & однородны \\
200 & 400 & 0.0275 & однородны \\
200 & 600 & 0.0257 & однородны \\
200 & 800 & 0.0243 & однородны \\
200 & 1000 & 0.0246 & однородны \\
400 & 600 & 0.0108 & однородны \\
400 & 800 & 0.0127 & однородны \\
400 & 1000 & 0.0132 & однородны \\
600 & 800 & 0.0123 & однородны \\
600 & 1000 & 0.0118 & однородны \\
800 & 1000 & 0.0050 & однородны \\
\hline
\end{tabular}
\end{table}

\paragraph{Вывод.}

При увеличении объёма выборок значения $D_{n_1, n_2}$ должны монотонно убывать, что отражает сходимость эмпирических функций распределения к истинной функции распределения. Все выборки должны быть однородны друг другу в асимптотическом смысле, поскольку они сгенерированы из одного распределения.

\newpage
\subsection{Проверка гипотез для реальных данных}

В данном разделе проводится проверка статистических гипотез о виде распределения для реальных данных, проанализированных в разделе 3.3 (дз), в данной работе раздел 3.4. Используются критерий Колмогорова-Смирнова и критерий хи-квадрат ($\chi^2$) Пирсона. Уровень значимости: $\alpha = 0.05$.

\subsubsection{Геометрическое распределение: данные о конверсии клиентов}

\paragraph{Источник данных.}

Используются данные UCI Online Retail Dataset (раздел 3.3): число покупок клиента до первой покупки, превышающей порог £8. Объём выборки $n = 500$, минимальное значение 1, максимальное 49.

\paragraph{Проверка простой гипотезы.}

\textbf{Гипотеза:}
\begin{equation}
H_0: \xi \sim \text{Geom}(\theta = 0.4) \quad \text{против} \quad H_1: \xi \not\sim \text{Geom}(\theta = 0.4)
\end{equation}

\textbf{Критерий Колмогорова-Смирнова.}

Статистика критерия:
\begin{equation}
D_n = \sup_x |F_n(x) - F_0(x)|
\end{equation}

где $F_n(x)$ --- эмпирическая функция распределения, $F_0(x) = 1 - (1-\theta)^{\lfloor x \rfloor}$ --- теоретическая функция распределения геометрического распределения.

\begin{table}[h]
\centering
\caption{Результаты критерия Колмогорова (простая гипотеза, геометрическое)}
\begin{tabular}{|l|c|}
\hline
\textbf{Характеристика} & \textbf{Значение} \\
\hline
Статистика $D_n$ & 0.4280 \\
Критическое значение $k_{0.95}/\sqrt{n}$ & 0.0608 \\
Вывод & $H_0$ \textbf{отвергается} \\
\hline
\end{tabular}
\end{table}

\textbf{Критерий хи-квадрат.}

Статистика критерия:
\begin{equation}
\chi^2 = \sum_{j=1}^k \frac{(\nu_j - np_j)^2}{np_j}
\end{equation}

где $\nu_j$ --- наблюдаемые частоты, $np_j$ --- ожидаемые частоты при $H_0$.

\begin{table}[h]
\centering
\caption{Результаты критерия $\chi^2$ (простая гипотеза, геометрическое)}
\begin{tabular}{|l|c|}
\hline
\textbf{Характеристика} & \textbf{Значение} \\
\hline
Статистика $\chi^2$ & 458.6966 \\
Число степеней свободы $\nu$ & 9 \\
Критическое значение $\chi^2_{0.95}(9)$ & 16.9190 \\
Вывод & $H_0$ \textbf{отвергается} \\
\hline
\end{tabular}
\end{table}

\paragraph{Проверка сложной гипотезы.}

\textbf{Гипотеза:}
\begin{equation}
H_0: \xi \sim \text{Geom}(\theta), \, \theta \text{ неизвестен} \quad \text{против} \quad H_1: \xi \not\sim \text{Geom}(\theta)
\end{equation}

Оценка параметра по данным: $\hat{\theta} = 1/\bar{X} = 0.4677$.

\begin{table}[h]
\centering
\caption{Результаты проверки сложной гипотезы (геометрическое)}
\begin{tabular}{|l|c|c|c|}
\hline
\textbf{Критерий} & \textbf{Статистика} & \textbf{Крит. знач.} & \textbf{Вывод} \\
\hline
Колмогоров-Смирнов & $D_n = 0.4677$ & 0.0608 & $H_0$ \textbf{отвергается} \\
Хи-квадрат & $\chi^2 = 387.90$, $\nu = 6$ & 12.5916 & $H_0$ \textbf{отвергается} \\
\hline
\end{tabular}
\end{table}

\textbf{Примечание:} При проверке сложной гипотезы число степеней свободы для критерия $\chi^2$ уменьшается на 1 (число оцениваемых параметров).

\paragraph{Вывод по геометрическому распределению.}

Оба критерия (Колмогорова-Смирнова и хи-квадрат) \textbf{отвергают} гипотезу о геометрическом распределении данных как при известном $\theta = 0.4$, так и при оценённом $\hat{\theta} = 0.47$. 

Это объясняется особенностями реальных данных:
\begin{itemize}
\item Наличие значительных выбросов (максимальное значение 49 при теоретическом ожидании около 2--3).
\item Высокая дисперсия данных ($S^2 = 18.2$), существенно превышающая теоретическую ($\mathbb{D}[\xi] = 3.75$).
\item Реальное поведение клиентов не полностью соответствует модели независимых испытаний Бернулли.
\end{itemize}

Несмотря на отвержение гипотезы, геометрическое распределение может использоваться как \textit{приближённая модель} для описания данных о конверсии, однако для точного моделирования требуются более сложные модели.

\newpage

\subsubsection{Нормальное распределение II: температурные данные}

\paragraph{Источник данных.}

Используются данные FiveThirtyEight Weather (раздел 3.3): среднесуточная температура в Хьюстоне за весенний период (март--май). Объём выборки $n = 92$, выборочное среднее $\bar{X} = 21.59$°C.

\paragraph{Проверка простой гипотезы.}

\textbf{Гипотеза:}
\begin{equation}
H_0: \xi \sim N(\mu = 22.5, \theta^2 = 16) \quad \text{против} \quad H_1: \xi \not\sim N(\mu = 22.5, \theta^2 = 16)
\end{equation}

\textbf{Критерий Колмогорова-Смирнова.}

\begin{table}[h]
\centering
\caption{Результаты критерия Колмогорова (простая гипотеза, нормальное)}
\begin{tabular}{|l|c|}
\hline
\textbf{Характеристика} & \textbf{Значение} \\
\hline
Статистика $D_n$ & 0.0972 \\
$p$-value & 0.3289 \\
Критическое значение $k_{0.95}/\sqrt{n}$ & 0.1418 \\
Вывод & $H_0$ \textbf{принимается} \\
\hline
\end{tabular}
\end{table}

\textbf{Критерий хи-квадрат.}

\begin{table}[h]
\centering
\caption{Результаты критерия $\chi^2$ (простая гипотеза, нормальное)}
\begin{tabular}{|l|c|}
\hline
\textbf{Характеристика} & \textbf{Значение} \\
\hline
Статистика $\chi^2$ & 10.3500 \\
Число степеней свободы $\nu$ & 4 \\
Критическое значение $\chi^2_{0.95}(4)$ & 9.4877 \\
Вывод & $H_0$ \textbf{отвергается} \\
\hline
\end{tabular}
\end{table}

\paragraph{Проверка сложной гипотезы.}

\textbf{Гипотеза:}
\begin{equation}
H_0: \xi \sim N(\mu, \theta^2), \, \mu, \theta \text{ неизвестны} \quad \text{против} \quad H_1: \xi \not\sim N(\mu, \theta^2)
\end{equation}

Оценки параметров по данным: $\hat{\mu} = 21.5882$, $\hat{\theta} = 4.7372$.

\begin{table}[h]
\centering
\caption{Результаты проверки сложной гипотезы (нормальное)}
\begin{tabular}{|l|c|c|c|}
\hline
\textbf{Критерий} & \textbf{Статистика} & \textbf{Крит. знач.} & \textbf{Вывод} \\
\hline
Колмогоров-Смирнов & $D_n = 0.1220$, $p = 0.119$ & 0.1418 & $H_0$ \textbf{принимается} \\
Хи-квадрат & $\chi^2 = 18.60$, $\nu = 3$ & 7.8147 & $H_0$ \textbf{отвергается} \\
\hline
\end{tabular}
\end{table}

\textbf{Примечание:} При проверке сложной гипотезы число степеней свободы для критерия $\chi^2$ уменьшается на 2 (число оцениваемых параметров).

\paragraph{Вывод по нормальному распределению.}

Критерии дают \textbf{противоречивые результаты}:

\begin{itemize}
\item \textbf{Критерий Колмогорова-Смирнова принимает} гипотезу о нормальности как для простой ($p = 0.33$), так и для сложной ($p = 0.12$) гипотез.
\item \textbf{Критерий хи-квадрат отвергает} гипотезу в обоих случаях.
\end{itemize}

Такое расхождение объясняется следующими факторами:
\begin{enumerate}
\item Критерий Колмогорова-Смирнова более чувствителен к отклонениям в центральной части распределения, тогда как критерий $\chi^2$ учитывает отклонения на хвостах.
\item Небольшой объём выборки ($n = 92$) и выбор числа интервалов влияют на мощность критерия $\chi^2$.
\item Температурные данные имеют слабую сезонную компоненту, которая нарушает строгую нормальность.
\end{enumerate}

В целом, \textbf{нормальное распределение является приемлемой моделью} для температурных данных, что подтверждается критерием Колмогорова-Смирнова. Отвержение критерием $\chi^2$ может быть связано с особенностями разбиения на интервалы и малым числом степеней свободы.

\subsubsection{Общий вывод}

Проведённый анализ реальных данных с использованием критериев проверки гипотез показал:

\begin{enumerate}
\item \textbf{Геометрическое распределение} (данные о конверсии клиентов): гипотеза о геометрическом распределении \textbf{отвергается} обоими критериями. Реальные данные имеют бо́льшую дисперсию и более тяжёлые хвосты, чем предсказывает модель. Для практических целей геометрическое распределение может использоваться как приближение, но точное моделирование требует более сложных моделей.

\item \textbf{Нормальное распределение II} (температурные данные): критерий Колмогорова-Смирнова \textbf{принимает} гипотезу о нормальности ($p > 0.05$), тогда как критерий $\chi^2$ её отвергает. Учитывая бо́льшую мощность критерия Колмогорова для непрерывных распределений, можно заключить, что нормальное распределение является \textbf{адекватной моделью} для температурных данных.

\item Использование реальных данных продемонстрировало, что теоретические модели распределений являются идеализацией, и реальные явления могут лишь приближённо описываться классическими распределениями.
\end{enumerate}

\end{document}
